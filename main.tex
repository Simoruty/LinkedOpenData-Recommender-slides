\documentclass{beamer}
\usepackage{beamerthemesplit}
\usepackage{booktabs}
\usepackage{graphicx}
\usepackage{transparent}
\usepackage[italian]{babel}
\usepackage[utf8x]{inputenc}
\usepackage{listings}
\usepackage{tikz}
\usetikzlibrary{arrows,shapes}
\tikzstyle{actor edge} = [red!90]
\tikzstyle{director edge} = [blue!90]

\usepackage{color}
\usepackage{xcolor}
\definecolor{var}{RGB}{20,105,176}
\colorlet{prefix}{magenta!60!black}
\colorlet{keyword}{red!60!black}
\lstdefinelanguage{customized}{
	basicstyle=\fontsize{8}{10}\selectfont \ttfamily,%
	columns=fullflexible,%
	showstringspaces=false,%
	breaklines=true,%
	frame=lines,%
	xleftmargin=10mm, %
	xrightmargin=25mm, %
	breakatwhitespace=false, %
	tabsize=2, %
	captionpos=b, %
	 literate=
	 *{PREFIX}{{{\color{keyword}{PREFIX }}}}{1}
	 {SELECT}{{{\color{keyword}{SELECT }}}}{1}
	 {FROM}{{{\color{keyword}{FROM }}}}{1}
	 {WHERE}{{{\color{keyword}{WHERE }}}}{1}
	 {?name}{{{\color{var}{?name}}}}{1}
	 {dbpedia: }{{{\color{prefix}{dbpedia:}}}}{1}
	 {prop: }{{{\color{prefix}{prop:}}}}{1}
}

\title[LOD CB-RS]{Linked Open Data per un Content-based Recommender System}
\institute{ \textbf{Accesso intelligente alle informazioni ed \\ elaborazione del linguaggio naturale\\}
~ \\
\begin{small}
Corso di Laurea in Informatica Magistrale
\end{small}}
\author{\textbf{Luciano Quercia}\\
\textbf{Simone Rutigliano}}
\date{\tiny{\today}}

\usebackgroundtemplate{
%    \transparent{0.12}{
        \includegraphics[width=\paperwidth, height=\paperheight]{./figure/escher_hands_tr.png}
%    }
}

%\usetheme{Hannover}
\usetheme{Copenhagen}
\usecolortheme{seahorse}
\usecolortheme{rose}
%\usetheme{Frankfurt}
%\usecolortheme{beetle}

%\useoutertheme[subsection=false]{smoothbars}
%\useoutertheme[subsection=false]{smoothtree}
\useoutertheme{shadow}
\setbeamercovered{dynamic}

\pgfdeclareimage[height=1cm]{logo}{figure/logo}
\logo{\pgfuseimage{logo}}



\usepackage{tikz}
\usetikzlibrary{arrows,shapes}
\tikzstyle{vertex}=[circle,fill=black!25,minimum size=20pt,inner sep=0pt]
\tikzstyle{selected vertex} = [vertex, fill=red!24]
\tikzstyle{edge} = [draw,thick,-]
\tikzstyle{weight} = [font=\small]
\tikzstyle{selected edge} = [draw,line width=5pt,-,red!50]
\tikzstyle{ignored edge} = [draw,line width=5pt,-,black!20]


\begin{document}

%%%%%%%%%%%%%%%%%%%%%%%%%%%%%%%%%%%%%%%%%%%%%%%%%%%%%

\begin{frame}
\maketitle
\end{frame}

%%%%%%%%%%%%%%%%%%%%%%%%%%%%%%%%%%%%%%%%%%%%%%%%%%%%%

\begin{frame}
\frametitle{Outline}
\tableofcontents
\end{frame}

%%%%%%%%%%%%%%%%%%%%%%%%%%%%%%%%%%%%%%%%%%%%%%%%%%%%%

\section{Obiettivi}
\begin{frame}
\frametitle{Obiettivi}
Realizzazione di un \textbf{content-based recommender system}

basato sulla \textbf{Linked Open Data Cloud}
\end{frame}

%%%%%%%%%%%%%%%%%%%%%%%%%%%%%%%%%%%%%%%%%%%%%%%%%%%%%

\begin{frame}
\frametitle{Content-based Recommender System}
Il sistema stabilisce a priori la distanza trai film al fine di raccomandare i più simili alle preferenze dell'utente

\begin{figure}
\includegraphics[width=.8\textwidth]{figure/cbrs.jpg}
\end{figure}

\end{frame}

%%%%%%%%%%%%%%%%%%%%%%%%%%%%%%%%%%%%%%%%%%%%%%%%%%%%%

\begin{frame}
\frametitle{Linked Open Data Cloud}

%\includegraphics[width=.95\textwidth]{figure/LoDLogo.png}

Collezione (\textbf{Cloud}) di dataset:
\begin{itemize}
\item descritti attraverso RDF
\item fortemente interconnessi fra loro (\textbf{Linked})
\item fruibili liberamente e gratuitamente (\textbf{Open})
\end{itemize}
\end{frame}

%%%%%%%%%%%%%%%%%%%%%%%%%%%%%%%%%%%%%%%%%%%%%%%%%%%%%

\begin{frame}
\frametitle{Linked Open Data Cloud}
\includegraphics[width=.95\textwidth]{figure/lodcloud}
\end{frame}

%%%%%%%%%%%%%%%%%%%%%%%%%%%%%%%%%%%%%%%%%%%%%%%%%%%%%

\begin{frame}
\frametitle{Resource Description Framework}
Strumento base proposto da \emph{W3C} per la codifica, lo scambio e il riutilizzo di metadati strutturati.

L'RDF Data Model si basa su tre principi chiave:
\begin{enumerate}
\item qualunque cosa può essere identificata da un (URI)
\item utilizzare il linguaggio meno espressivo per definire qualunque cosa
\item qualunque cosa può dire qualunque cosa su qualunque cosa
\end{enumerate}
\end{frame}

%%%%%%%%%%%%%%%%%%%%%%%%%%%%%%%%%%%%%%%%%%%%%%%%%%%%%

\begin{frame}
\frametitle{Esempio - Resource Description Framework}
Considerando la frase:\\~\\
\begin{center} \emph{Tarantino is the director of the Django Unchained.} \\~\\
\end{center}
L'affermazione può essere suddivisa come: \\~\\
\begin{tabular}{ l | l }
    Soggetto (Risorsa) & Django Unchained \\
    Predicato (Proprietà) & director \\
    Oggetto (Risorsa) & Tarantino \\
\end{tabular}
\end{frame}

%%%%%%%%%%%%%%%%%%%%%%%%%%%%%%%%%%%%%%%%%%%%%%%%%%%%%

\section{Progetto}

\subsection{Sorgente dati}

\begin{frame}
\frametitle{DBPedia}
\begin{columns}

\begin{column}{0.4\textwidth}
\vspace{1cm}
\centering\includegraphics[width=.8\textwidth]{figure/dbpedialogo}
\begin{itemize}
\item Centro della Linked Open Data Cloud
\item Dump di Wikipedia trasformato in RDF
\end{itemize}
\end{column}
\begin{column}{0.6\textwidth}
\includegraphics[width=1\textwidth]{figure/AboutDBPedia}
\end{column}
\end{columns}
\end{frame}

\begin{frame}
\frametitle{Proprietà estratte}
Per la raccomandazione di film, abbiamo estratto le seguenti proprietà
\begin{columns}
\begin{column}{0.5\textwidth}
\begin{itemize}
\item studio
\item music
\item music composer
\item writer
\item editing
\item director
\end{itemize}
\end{column}
\begin{column}{0.5\textwidth}
\begin{itemize}
\item subject
\item starring
\item productor
\item writer
\item cinematography
\end{itemize}
\end{column}
\end{columns}
\end{frame}

%%%%%%%%%%%%%%%%%%%%%%%%%%%%%%%%%%%%%%%%%%%%%%%%%%%%%

\subsection{Realizzazione}

%%%%%%%%%%%%%%%%%%%%%%%%%%%%%%%%%%%%%%%%%%%%%%%%%%%%%

\begin{frame}[fragile]
\frametitle{Grafo delle Risorse}
Attraverso query SPARQL sono state estratte tutte le triple che avevano proprietà nota e un film come soggetto
è stato generato il grafo delle risorse \\~\\
\begin{lstlisting}[language=customized]
PREFIX dbpedia: http://dbpedia.org/resource/
PREFIX prop: http://dbpedia.org/ontology/
SELECT ?name
WHERE {
dbpedia:Django_Unchained prop:director ?name .
}
\end{lstlisting}
risultato:

\begin{lstlisting}[language=customized]
http://dbpedia.org/resource/Quentin_Tarantino
\end{lstlisting}
\end{frame}

%%%%%%%%%%%%%%%%%%%%%%%%%%%%%%%%%%%%%%%%%%%%%%%%%%%%%

\begin{frame}
\frametitle{Grafo delle Risorse}
\begin{columns}[p]
\begin{column}{0.8\textwidth}
\begin{tikzpicture}
[lineDecorate/.style={-,thick},%
  actor/.style={shape=circle,inner sep=3pt,draw, thick},
  film/.style={shape=diamond,inner sep=4pt,draw,thick, fill=blue!15}]

%% nodes or vertices
\foreach \nodename/\x/\y/\direction/\navigate in {
  $Django Unchained$/1/3/left/west, $Inglourious Basterds$/3/3/right/east, $Titanic$/1/1/below/south}
{
  \node (\nodename) at (\x,\y) [film] {};
  \node [\direction] at (\nodename.\navigate) {\footnotesize$\nodename$};
}

\foreach \nodename/\x/\y/\direction/\navigate in {
  $Quentin Tarantino$/2/4/above/north, $Leonardo Di caprio$/1/2/left/west, $Christoph Waltz$/3/2/right/east}
{
  \node (\nodename) at (\x,\y) [actor] {};
  \node [\direction] at (\nodename.\navigate) {\footnotesize$\nodename$};
}

%% edges or lines
\path
\foreach \startnode/\endnode in {$Django Unchained$/$Leonardo Di caprio$, $Django Unchained$/$Christoph Waltz$, $Inglourious Basterds$/$Christoph Waltz$, $Leonardo Di caprio$/$Titanic$}
{
  (\startnode) edge[lineDecorate,director edge] node {} (\endnode)
}

\foreach \startnode/\endnode in {$Django Unchained$/$Quentin Tarantino$,$Quentin Tarantino$/$Inglourious Basterds$}
{
  (\startnode) edge[lineDecorate,bend left,actor edge] node {} (\endnode)
}

\foreach \startnode/\endnode in {$Quentin Tarantino$/$Django Unchained$, $Inglourious Basterds$/$Quentin Tarantino$}
{
  (\startnode) edge[lineDecorate,bend left,director edge] node {} (\endnode)
};

\end{tikzpicture}
\end{column}

\begin{column}{0.35\textwidth}
\begin{tabular}{l l}
  \begin{tikzpicture}[film/.style={shape=diamond,inner sep=3pt,draw,thick, fill=blue!15}]
    \node (a) at (6,3) [film] {};
    \node [right] at (a.west) {\footnotesize};
  \end{tikzpicture}
  & Film \\

 \begin{tikzpicture}[actor/.style={shape=circle,inner sep=3pt,draw, thick}]
    \node (a) at (9,3) [actor] {};
    \node [right] at (a.west) {\footnotesize};
 \end{tikzpicture}
 & Resource \\

  \color{red}--- & Edge director \\
 \color{blue}--- & Edge actor \\
\end{tabular}

\end{column}
\end{columns}
\end{frame}

%%%%%%%%%%%%%%%%%%%%%%%%%%%%%%%%%%%%%%%%%%%%%%%%%%%%%

\begin{frame}
\frametitle{Grafo dei Film}
Tutte le risorse non film sono state epurate ed inglobate all'interno degli archi.
\\~\\
\begin{tikzpicture}
[lineDecorate/.style={-,thick},%
  film/.style={shape=diamond,inner sep=4pt,draw,thick, fill=blue!15}]

%% nodes or vertices
\foreach \nodename/\x/\y/\direction/\navigate in {
  $Django Unchained$/0/4/left/west, $Inglourious Basterds$/4/4/right/east, $Titanic$/0/1/below/south}
{
  \node (\nodename) at (\x,\y) [film] {};
  \node [\direction] at (\nodename.\navigate) {\small$\nodename$};
}
  \node ($actor Di Caprio actor$) at (0,2) {\color{red}\scriptsize{actor Di Caprio actor}};

%% edges or lines
\path
\foreach \startnode/\endnode in {$Django Unchained$/$actor Di Caprio actor$,$actor Di Caprio actor$/$Titanic$}
{
  (\startnode) edge[lineDecorate,actor edge] node[] {} (\endnode)
}

\foreach \startnode/\endnode in {$Django Unchained$/$Inglourious Basterds$}
{
  (\startnode) edge[lineDecorate,bend left,actor edge] node[auto] {\scriptsize{actor Tarantino actor}} (\endnode)
}

\foreach \startnode/\endnode in {$Inglourious Basterds$/$Django Unchained$}
{
  (\startnode) edge[lineDecorate,bend left,director edge] node[auto] {\scriptsize{director Tarantino director}} (\endnode)
};

\end{tikzpicture}
\end{frame}

%%%%%%%%%%%%%%%%%%%%%%%%%%%%%%%%%%%%%%%%%%%%%%%%%%%%%

\subsection{Fattori}

%%%%%%%%%%%%%%%%%%%%%%%%%%%%%%%%%%%%%%%%%%%%%%%%%%%%%

\begin{frame}
\frametitle{Distanze}
Sono state applicate 4 distanze su grafo:

\begin{itemize}
\item<1-> Direct
\item<2-> Combinated
\item<3-> Direct Weighted
\item<4-> Combinated Weighted
\end{itemize}
\end{frame}

%%%%%%%%%%%%%%%%%%%%%%%%%%%%%%%%%%%%%%%%%%%%%%%%%%%%%

\begin{frame}
\frametitle{Rappresentazione del profilo}
Il profilo è stato rappresentato in 2 modi:
\begin{center}
\begin{itemize}
	\item \textsc{Simple} - Insieme di film positivi per l'utente
	\pause
	\item \textsc{Weighted} - Ogni film influisce, positivamente o negativamente, alle raccomandazioni, secondo il voto ricevuto
\end{itemize}
\pause
$$
P_{NORM}(f_a) = Voto(f_a)- Voto_M\footnote{Valor medio delle votazioni pari a 3}
$$
\end{center}

\end{frame}

%%%%%%%%%%%%%%%%%%%%%%%%%%%%%%%%%%%%%%%%%%%%%%%%%%%%%

\begin{frame}
	\frametitle{Esempio di profilazione}
	considerati i film con le relative votazioni:
	
\begin{center}	
	\begin{tabular}{l c}
		\multicolumn{1}{c}{\textsc{Film}} & \textsc{Votazione} \\
		\hline
		Titanic & 5 \\
		Django Unchained & 4 \\
		Bastardi senza gloria & 2 \\
		\hline
	\end{tabular}
\end{center}	
\medskip
I profili creati saranno:
\begin{center}
\begin{tabular}{l l}
	\textbf{Simple} &
	\begin{tabular}{l}
		Django Unchained \\		
		Titanic \\
	\end{tabular} \\
	\\
	
	\textbf{Weighted} &
	\begin{tabular}{l r}
		%		Film & rilevanza \\
		Titanic & 2 \\
		Django Unchained & 1 \\
		Bastardi senza gloria & -1\\
	\end{tabular}	\\
\end{tabular}
\end{center}	
\end{frame}

%%%%%%%%%%%%%%%%%%%%%%%%%%%%%%%%%%%%%%%%%%%%%%%%%%%%%

\subsection{Output}

%%%%%%%%%%%%%%%%%%%%%%%%%%%%%%%%%%%%%%%%%%%%%%%%%%%%%

\begin{frame}
\frametitle{Raccomandazioni}
\end{frame}

%%%%%%%%%%%%%%%%%%%%%%%%%%%%%%%%%%%%%%%%%%%%%%%%%%%%%

\section{Sperimentazione}
\subsection{Dataset}

%%%%%%%%%%%%%%%%%%%%%%%%%%%%%%%%%%%%%%%%%%%%%%%%%%%%%

\begin{frame}
\frametitle{MovieLens}
\end{frame}

%%%%%%%%%%%%%%%%%%%%%%%%%%%%%%%%%%%%%%%%%%%%%%%%%%%%%

\subsection{Protocollo Sperimentale}

%%%%%%%%%%%%%%%%%%%%%%%%%%%%%%%%%%%%%%%%%%%%%%%%%%%%%

\begin{frame}
\frametitle{Protocollo Sperimentale}
\end{frame}

%%%%%%%%%%%%%%%%%%%%%%%%%%%%%%%%%%%%%%%%%%%%%%%%%%%%%

\begin{frame}
\frametitle{Metriche}
\end{frame}

%%%%%%%%%%%%%%%%%%%%%%%%%%%%%%%%%%%%%%%%%%%%%%%%%%%%%

\subsection{Risultati}

%%%%%%%%%%%%%%%%%%%%%%%%%%%%%%%%%%%%%%%%%%%%%%%%%%%%%

\begin{frame}
\frametitle{Risultati}
\end{frame}

%%%%%%%%%%%%%%%%%%%%%%%%%%%%%%%%%%%%%%%%%%%%%%%%%%%%%

\section{Conclusioni e sviluppi futuri}
\begin{frame}
\frametitle{Conclusioni e sviluppi futuri}
\end{frame}

%%%%%%%%%%%%%%%%%%%%%%%%%%%%%%%%%%%%%%%%%%%
%%%%%%%%%%%%%%%%%%%%%%%%%%%%%%%%%%%%%%%%%%%
%%%%%%%%%%%%%%%%%%%%%%%%%%%%%%%%%%%%%%%%%%%


\begin{frame}
\begin{center}
Grazie per l'attenzione.
\end{center}
\end{frame}
\end{document}
